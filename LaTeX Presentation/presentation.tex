\documentclass{beamer}

\mode<presentation>
{
  \usetheme{AnnArbor} %Copenhagen
  \usecolortheme{wolverine}
  \setbeamercovered{transparent}
}

\usepackage[english]{babel}
\usepackage[latin1]{inputenc}
\usepackage{times}
\usepackage[T1]{fontenc} 
% Or whatever. Note that the encoding and the font should match. If T1
% does not look nice, try deleting the line with the fontenc.
\usepackage{amsmath}

\newcommand{\linespace}{\vskip 0.25cm}

\definecolor{MyForestGreen}{rgb}{0,0.7,0} 
\newcommand{\tableemph}[1]{{#1}}
\newcommand{\tablewin}[1]{\tableemph{#1}}
\newcommand{\tablemid}[1]{\tableemph{#1}}
\newcommand{\tablelose}[1]{\tableemph{#1}}

\definecolor{MyLightGray}{rgb}{0.6,0.6,0.6}
\newcommand{\tabletie}[1]{\color{MyLightGray} {#1}}

% The text in square brackets is the short version of your title and will be used in the
% header/footer depending on your theme.
\title[Morphology in Art Restoration]{Morphological Operations Applied to \\ Digital Art Restoration}

% Sub-titles are optional - uncomment and edit the next line if you want one.
% \subtitle{Why does sub-tree crossover work?} 

% The text in square brackets is the short version of your name(s) and will be used in the
% header/footer depending on your theme.
\author[Dramdahl]{M. Kirbie Dramdahl}

% The text in square brackets is the short version of your institution and will be used in the
% header/footer depending on your theme.
\institute[U of Minn, Morris]
{
  Division of Science and Mathematics \\
  University of Minnesota, Morris \\
  Morris, Minnesota, USA
}

% The text in square brackets is the short version of the date if you need that.
\date[April '14, Sen. Sem., UMM] % (optional)
{29 April 2014 \\ UMM CSci Senior Seminar Conference \\ University of Minnesota, Morris}

% Delete this, if you do not want the table of contents to pop up at
% the beginning of each subsection:
\AtBeginSection[]
{
  \begin{frame}<beamer>
    \frametitle{Outline}
    \tableofcontents[currentsection, hideothersubsections]
  \end{frame}
}

\begin{document}

\begin{frame}
  \titlepage
\end{frame}

% For a 20-25 minute senior seminar talk you probably want something like:
% - Two or three major sections (other than the summary).
% - At *most* three subsections per section.
% - Talk about 30s to 2min per frame. So there should probably be between
%   15 and 30 frames, all told.

\section*{Overview}

\subsection*{Outline}

\begin{frame}
  \frametitle{Outline}
  \tableofcontents[hideallsubsections]
\end{frame}

\section[Edge Detection]{Edge Detection}

\section[Morphological Operations]{Morphological Operations}

\subsection[Erosion]{Erosion}

\subsection[Dilation]{Dilation}

\subsection[Opening]{Opening}

\subsection[Closing]{Closing}

\section[Methods of Crack Detection]{Methods of Crack Detection}

\subsection[Top-Hat Transform]{Top-Hat Transform}

\subsubsection[White Top-Hat]{White Top-Hat}

\subsubsection[Black Top-Hat]{Black Top-Hat}

\subsubsection[Multiscale Top-Hat]{Multiscale Top-Hat}

\subsection[Alternative Method]{Alternative Method}

\section[Inpainting]{Inpainting}

\section[Results]{Results}

\subsection[Top-Hat Transform Results]{Top-Hat Transform Results}

\subsection[Alternative Method Results]{Alternative Method Results}

\section[Conclusions]{Conclusions}

\begin{frame}
	\frametitle{Thanks!}
	
	Thank you for your time and attention!
		
	\linespace
	\linespace
	
	Contact: \texttt{dramd002@morris.umn.edu}
	
	\linespace
	\linespace
	
	\begin{center}
	{\huge Questions?}
	\end{center}
\end{frame}

\section*{References}

\begin{frame} 
	\frametitle{References} 
	
	\begin{thebibliography}{lskdjf}
	\end{thebibliography}
\end{frame} 

\end{document}